



%%%%%%%%%%%%%%%%%%%%%%%%%%%%%%%%%%%%%%%%%%%%




We end this section by explaining the origin of the functions
$\mathcal{P}_\ast$. Let
$\eps(x)$ denote the error on $x\in\mathsf{Real}_{u,p}$. We have $\eps(x)< 2^{u-p+1}$
 The errors on the addition and product may be bounded
by $e_+=\eps(x)+\eps(y)$ and $e_\times=y\cdot\eps(x)+x\cdot \eps(y)+ \eps(x)\cdot \eps(y)$,
respectively. Then the most significant bits of the errors have weights $\mathsf{ufp}(e_+)$ and $\mathsf{ufp}(e_\times)$
and the accuracies of the results are $\mathsf{ufp}(x + y)-\mathsf{ufp}(e_+)$ and $\mathsf{ufp}(x \times y)-\mathsf{ufp}(e_\times)$,
respectively.
We need to over-approximate $e_+$ in order to ensure $p$. 
We have
$
\eps(x)< 2^{u_1-p_1+1}$ and $\eps(y)< 2^{u_2-p_2+1}
$
and, consequently,
$
e_+ < 2^{u_1-p_1+1}+  2^{u_2-p_2+1}.
$
We introduce the function $\iota(x,y)$ also
defined in Figure \ref{figtypprim} and which is equal to $1$ if $x=y$ and $0$ otherwise.
 We have
$$
\begin{array}{rcl}
\mathsf{ufp}(e_+) &<& \max(u_1-p_1+1,u_2-p_2+1)+\iota(u_1-p_1,u_2-p_2) \\
          &\le& \max(u_1-p_1,u_2-p_2)+\iota(u_1-p_1,u_2-p_2)
\end{array}$$
and we conclude that 
\begin{equation}
p = u-\max(u_1-p_1,u_2-p_2)-\iota(u_1-p_1,u_2-p_2)\enspace .
\end{equation}
For the forward product, we have
$p	= \mathsf{ufp}(x\times y)-\mathsf{ufp}(e_\times)$ with $e_\times=x\cdot \eps(y)+y\cdot \eps(x)+\eps(x_\cdot\eps(y)$. 
We have, by definition of $\mathsf{ufp}$,
$
2^{u_1} \le x < 2^{u_1+1}\quad\text{and}\quad 2^{u_2} \le y< 2^{u_2+1}\enspace.
$
Then $e_\times$ may be bound by
$$
\begin{array}{rcl}
e_\times &<& 2^{u_1+1}\cdot 2^{u_2-p_2+1} + 2^{p_2+1}\cdot 2^{u_1-p_1+1}+ 2^{u_1-p_1+1}\cdot 2^{u_2-p_2+1}\\
            &=& 2^{u_1+u_2-p_2+2}+ 2^{u_1+u_2-p_1+2}+ 2^{u_1+u_2-p_1-p_2+2} \enspace.
\end{array}
$$
Since $u_1+u_2-p_1-p_2+2< u_1+u_2-p_1+2$ and $u_1+u_2-p_1-p_2+2< u_1+u_2-p_2+2$, 
we may get rid of the last term of the former equation and we
obtain that
$$
\begin{array}{rcl}
\mathsf{ufp}(e_\times) &<& \max(u_1+u_2-p_1+2,u_1+u_2-p_2+2)+\iota(p_1,p_2)\\
&\le & \max(u_1+u_2-p_1+1,u_1+u_2-p_2+1)+\iota(p_1,p_2)\enspace.
\end{array}
$$
We conclude that
\begin{equation}
p = u-\max(u_1+u_2-p_1+1,u_1+u_2-p_2+1)-\iota(p_1,p_2)\enspace .
\end{equation}
Note that, by reasoning on the exponents of the values, the constraints resulting from a product become linear.
The equations for the subtraction and product are identical to the equations for the addition and product, 
respectively.


