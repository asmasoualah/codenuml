
\section{Programming with Types for Numerical Accuracy}
\label{over}

In this section, we present informally how our type system works
throughout a programming sequence in our language, \texttt{Numl}.
First of all, we use real 
numbers  $\mathtt{r}\{s,u,p\}$ where $\mathsf{r}$
is the value itself, and $\{s,u,p\}$ the format of $\mathsf{r}$.
The format or a real number is made of a sign $s\in \mathsf{Sign}$ and integers 
$u,p\in \mathsf{Int}$ such that $u$ is the unit in the first place of $\mathsf{r}$,
written $\mathsf{ufp}(\mathsf{r})$ and $p$
the precision.

We have $\mathsf{Sign}=\{0,\oplus,\ominus,\top\}$ and $\mathsf{sign(r)}=0$ if $\mathsf{r}=0$,
$\mathsf{sign(r)}=\oplus$ if $\mathsf{r}>0$ and $\mathsf{sign(r)}=\ominus$ if $\mathsf{r}<0$.
The set $\mathsf{Sign}$ is equipped with the partial order relation $\prec\subseteq \mathsf{Sign}
\times \mathsf{Sign}$ defined by $0\prec \oplus$, $0\prec \ominus$, $\oplus\prec \top$ and 
$\ominus\prec \top$. 

The $\mathsf{ufp}$ of a number $x$ is 
\begin{equation}\label{equfp}
\mathsf{ufp}(x) = \min \big\{ i\in \mathbb{N}\ :\ 2^{i+1} > x\big\} = \lfloor \log_2(x)\rfloor\enspace.
\end{equation}

The term $p$ defines the precision of $\mathsf{r}$. Let $\eps(\mathsf{r})$
be the absolute error on $\mathsf{r}$, we assume that
\begin{equation}
\eps(\mathsf{r})< 2^{u-p+1}\enspace .
\end{equation}
The errors on the numerical constants arising in programs are specified by the user or determined by default by 
the system. The errors on the computed values can be inferred by  propagation of the
initial errors.
Similarly to Equation (\ref{equfp}), we also define the
 \textit{u}nit in the \textit{l}ast \textit{p}lace ($\mathsf{ulp}$) used later in this article.
The $\mathsf{ulp}$ of a number of precision $p$ is defined by
\begin{equation}
 \mathsf{ulp}(x) = \mathsf{ufp}(x) - p+1\enspace .
\end{equation}

\begin{figure}[tb]
\small
\hrule
\vspace{0.2cm}
\centerline{
\begin{tabular}{cccccccccccccccccc}
&Format    &&& Name &&& $p$ &&& $e$ bits  &&& $e_{min}$ &&& $e_{max}$ & \\
\hline
&&&&&&&&&&&&&&&&&\\
&Binary16  &&& Half precision            &&& 11  &&& 5              &&& $-14 $  &&& $+15$ &     \\
&Binary32  &&& Single precision            &&& 24  &&& 8              &&& $-126 $  &&& $+127$ &     \\
&Binary64  &&& Double precision           &&& 53  &&& 11             &&& $-1122 $  &&& $+1223$&     \\
\hspace{0.05cm} & Binary128 &  \hspace{0.05cm}  &  \hspace{0.05cm}  & Quadruple precision            
& \hspace{0.05cm}  &  \hspace{0.05cm}  &113  & \hspace{0.05cm} &  \hspace{0.05cm} & 15             
& \hspace{0.05cm}  & \hspace{0.05cm} & $-16382 $  & \hspace{0.05cm}  & \hspace{0.05cm} & $+16383$ 
&  \hspace{0.05cm}      \\
\end{tabular}}\vspace{0.2cm}
\hrule
\normalsize
\caption{\label{formats}Basic binary IEEE754  formats.}
\end{figure}


For example, the type of $1.234$ is $\F{+}{0}{53}$ since $\mathsf{ufp}(1.234)=0$ and since we assume that, 
by default, the real
numbers have the same precision than in the IEEE754 double precision floating-point format \cite{IEEE754} 
(see Figure \ref{formats}). 
Other formats may be specified by the programmer, as in the example below.
Let us also mention that our type system is independent of a given computer arithmetic. The 
interpreter only needs to implement the formats given by the type system, using floating-point numbers, fixed-point
numbers \cite{MENTOR}, multiple precision numbers\footnote{\tt https://gmplib.org/}, etc
in order to ensure that the finite precision operations are computed exactly.
The special case of the IEEE754 floating-point arithmetic, which introduces additional
errors due to the roundoff of the results of the operations is discussed in Section \ref{ieee}. 
\begin{verbatim}
> 1.234 ;;
- : real{+,0,53} = 1.234000000000000

> 1.234{4};;
- : real{+,0,4} = 1.2
\end{verbatim}
Remark that in our implementation the type information is used by the pretty printer to display 
only the correct digits of a number and a bound on the roundoff error.

Note that
accuracy is not a property of a number but  a number that states how closely a particular floating-point number
matches some ideal true value.
For example, using the basis $\beta=10$ for the sake of simplicity, the floating-point value $3.149$ represents $\pi$ with an accuracy of $3$.
It itself has a precision of $4$. It represents the real number $3.14903$
with an accuracy of $4$. 


As in \texttt{ML}, our type system admits parameterized types \cite{Pie02}. 
\begin{verbatim}
> let f = fun x -> x + 1.0 ;;
val f : real{'a,'b,'c} -> real{<expr>,<expr>,<expr>} = <fun>

> verbose true ;;
- : unit = ()

> f ;;
- : real{'a,'b,'c} -> real{(SignPlus 'a 'b 1 0),
\end{verbatim}
\ \hspace{4.55cm}\texttt{((max 'b 0) +\_ (sigma+ 'a 1)),}

~ \hspace{4.2cm}\texttt{((((max 'b 0) +\_ (sigma+ 'a 1)) }

~ \hspace{4.6cm}\texttt{-\_ (max ('b -\_ 'c) -53))} 

~ \hspace{4.2cm}\texttt{-\_ (iota ('b -\_ 'c) -53))\} = <fun>}

\smallskip

In the example above, the type of \texttt{f} is a function of an argument whose parameterized type is $\F{\texttt{'a}}{\texttt{'b}}{\texttt{'c}}$, where
\texttt{'a}, \texttt{'b} and \texttt{'c} are three type variables. The return type of the function
 \texttt{f} is \texttt{Real\{}$e_0$\texttt{,}$e_1$\texttt{,}$e_2$\texttt{\}} 
where $e_0$, $e_1$ and $e_2$ are arithmetic expressions
containing the variables \texttt{'a}, \texttt{'b} and \texttt{'c}. By default these expressions are not displayed by the system (just like higher order values are not explicitly
displayed in \texttt{ML} implementations) but we may enforce the system to print them. In our implementation we write \texttt{+}, \texttt{-}, \texttt{*} and
\texttt{/} the floating-point operators and \texttt{+\_}, \texttt{-\_}, \texttt{*\_} and
\texttt{/\_} their integer versions. The expressions arising in the type of \texttt{f} are explained in Section \ref{infe}. As shown below, various
applications of \texttt{f} yield results of various types, depending on the type of the argument.
\begin{verbatim}
> f 1.234 ;;
- : real{+,1,53} = 2.234000000000000

> f 1.234{4} ;;
- : real{+,1,5} = 2.2
\end{verbatim}

If the interpreter detects that the result of some computation has no significant digit, then an error is raised. For example, it is well-known that
in IEEE754 double precision $(10^{16} + 1)-10^{16}=0$. Our type system rejects this computation.
\begin{verbatim}
> (1.0e15 + 1.0) - 1.0e15 ;;
- : real{+,50,54} = 1.0 

> (1.0e16 + 1.0) - 1.0e16 ;;
Error: The computed value has no significant digit. Its ufp is 0 but the ulp of the 
certified value is 1 
\end{verbatim}

Last but not least, our type system accepts recursive functions. For example, we have:
\begin{verbatim}
> let rec g x = if x < 1.0 then x else g (x * 0.07) ;;
val g : real{+,0,53} -> real{+,0,53} = <fun>

> g 1.0 ;;
- : real{+,0,53} = 0.07000000000000

> g 2.0 ;;
Error: This expression has type real{+,1,53} but an expression was expected of type
real{+,0,53}
\end{verbatim}

In the above session, the type system unifies the return type of the function with the
type of the conditional. The types of the \texttt{then} and \texttt{else} branches also need to be unified.
Then the return type is \texttt{real\{+,0,53\}} which corresponds to the type of the
value \texttt{1.0} used in the \texttt{then} branch. The type system also unifies the return
type with the type of the argument since the function is recursive. Finally, we obtain that the type
of \texttt{g} is \texttt{real\{+,0,53\} -> real\{+,0,53\}}.
As a consequence, we cannot call \texttt{g}  with an argument whose \textsf{ufp} is greater
than $\mathsf{ufp}(1.0)=0$.
To overcome this limitation, we introduce new comparison operations for real numbers. While
the standard comparison operator \texttt{<} has type \texttt{'a -> 'a -> bool},
the operator \texttt{<\{s,u,p\}} has type \texttt{real\{s,u,p\} -> real\{s,u,p\} -> bool}.
In other words, the compared value are cast in the format $\{s,u,p\}$ before performing the
comparison. Now we can write the code:
\begin{verbatim}
> let rec g x = if x <{*,10,15} 1.0 then x else g (x * 0.07) ;;
val g : real{*,10,15} -> real{*,10,15} = <fun>

> g 2.0 ;;
- : real{*,10,15} = 0.1

> g 456.7 ;;
- : real{*,10,15} = 0.1

> g 4567.8 ;;
Error: This expression has type real{+,12,53} but an expression was expected of 
type real{*,10,15}
\end{verbatim}

Interestingly unstable functions (in the sense that the initial errors grow with 
the number of iterations) are not typable. This is a desirable property of our type system.
\begin{verbatim}
> let rec h n = if (n=0) then 1.0 else 3.33 * (h (n -_ 1)) ;;
Error: This expression has type real{+,-1,-1} but an expression was expected of 
type real{+,-3,-1}
\end{verbatim}

Stable computations should be always accepted by our type system. Obviously,
this is not the case and, as any non-trivial type system, our type system rejects some correct programs.
The challenge is then to accept enough programs to be useful from an end-user point of view.
We end this section by showing a more significant example representative of what our type
system accepts. More examples are given later in this article, in Section \ref{moreex}.

\begin{verbatim}
> let deriv f x h = ((f (x + h)) - (f x)) / h ;;
val deriv : (real{<expr>,<expr>,<expr>} -> real{'a,'b,'c}) 
             -> real{<expr>,<expr>,<expr>} -> real{'d,'e,'f} 
             -> real{<expr>,<expr>,<expr>} = <fun>

> let g x = (x*x) - (5.0*x) + 6.0 ;;
val g : real{'a,'b,'c} -> real{<expr>,<expr>,<expr>} = <fun>

> deriv g 2.0 0.01 ;;
- : real{*,5,51} = -0.9900000000000

> let gprime x = deriv g x 0.01 ;;
val gprime : real{<expr>,<expr>,<expr>} -> real{<expr>,<expr>,<expr>} = <fun>

> let newton x xold f fprime = if ((x-xold)<0.01{*,10,20}) then x 
                               else newton (x-((f x)/(fprime x))) x f fprime ;;
val newton : real{*,10,21} -> real{0,10,20} -> (real{*,10,21} -> real{'a,'b,'c})
             -> (real{*,10,21} -> real{'d,'e,'f}) -> real{*,10,21} = <fun>

> newton 9.0 0.0 g gprime ;;
- : real{*,10,21} = 5.771
\end{verbatim}

%In the programming session above, the function \texttt{rectangle}  implements the rectangle method 
%for numerical integration. It computes the integral $I=\int_a^b f(x)dx$ for a function $f$
%given as parameter (hence \texttt{rectangle} is an higher order function).
In the programming session above, we first define a higher order function \texttt{deriv}
which takes as argument a function and computes its numerical derivative at a given point.
Then we define a function \texttt{g} and compute the value of its derivative at point $2.0$.
Next, by partial application, we build a function computing the derivative of \texttt{g} at any point.
Finally, we define a function \texttt{newton} implementing Newton-Raphson method to find the
zero of a function. The \texttt{newton} function is also an higher order function taking as
argument the function for which a zero has to be found and its derivative. We call the
\texttt{newton} function with our function \texttt{g} and its derivative computed by partial application of the
\texttt{deriv} function. As a result we obtain a root of our polynomial \texttt{g} with a
guaranteed accuracy. Note that while Newton-Raphson method converges quadratically in the exact arithmetic,
numerical errors may perturb the process \cite{DMC15b}.

Obviously, our type system computes the propagation of the errors due to finite precision
but does not take care of the method error intrinsic to the implemented algorithm (Newton-Raphson method
in our case.)
All the programming sessions introduced above are fully interactive in our system, \texttt{Numl},
i.e. the type judgments are obtained instantaneously (about 0.01 second in average following our measurements)
including the most complicated ones of the last session.

