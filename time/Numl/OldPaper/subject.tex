
\begin{figure}[tb]
\hrule
\vspace{0.1cm}
$$
\frac{|\mathtt{r}-v_\bbf|< 2^{u-p+1} \quad \mathsf{ufp}(\mathtt{r})\le u\quad \mathsf{sign}(v_\bbf)\prec s}
     {\mathtt{r\{s,u,p\} } \rightarrow_\bbf v_\bbf}\quad\textsf{(FVal)}
\hspace{1cm}
\frac{v_\bbr=\mathtt{r}}
     {\mathtt{r\{s,u,p\} } \rightarrow_\bbr v_\bbr}\quad\textsf{(RVal)}
$$

$$
\frac{e_0 \rightarrow e_0'}
     {e_0\ast e_1 \rightarrow e_0'\ast e_1}\quad\textsf{(Op1)}
\hspace{1cm}
\frac{e_1 \rightarrow e_1'}
     {v\ast e_1 \rightarrow v\ast  e_1'}\quad\textsf{(Op2)}\hspace{1cm}\ast\in\{+,-,\times,\div,+\_,-\_,\times\_,\div_\_\}
$$


$$
\frac{v=v_0\ast v_1}
     {v_0\ast v_1 \rightarrow v}\quad\textsf{(Op)}\hspace{1cm}\ast\in\{+,-,\times,\div,+\_,-\_,\times\_,\div_\_\}
$$



$$
\frac{e_0 \rightarrow e_0'}
     {e_0\Join e_1 \rightarrow e_0'\Join e_1}\quad\textsf{(Cmp1)}
\hspace{1cm}
\frac{e_1 \rightarrow e_1'}
     {v\Join e_1 \rightarrow v\Join e_1'}\quad\textsf{(Cmp2)}
\hspace{1cm}\Join\in\{<_{\{u,p\}},>_{\{u,p\}},<,>\}
$$

$$
\frac{ b=(v_0^\bbf- v_1^\bbf\Join 2^{u-p+1})
}
     {v_0\Join_{\{u,p\}} v_1 \rightarrow_\bbf b}\quad\textsf{(FCmp)}
\hspace{1cm}%\Join\in\{<_{\{u,p\}},>_{\{u,p\}}\}
%$$
%$$
\frac{b=(v_0\Join v_1)
}
     {v_0\Join_{\{u,p\}} v_1 \rightarrow_\bbr b}\quad\textsf{(RCmp)}
\hspace{1cm}\Join\in\{<_{\{u,p\}},>_{\{u,p\}}\}
$$

$$
\frac{b=v_0\Join v_1}
     {v_0\Join v_1 \rightarrow b}\quad\textsf{(ACmp)}
\hspace{1cm}\Join\in\{=,\not=,<,>,\le,\ge\}
$$


$$
\frac{e_1 \rightarrow e_1'}
     {e_0\ e_1 \rightarrow e_0\ e_1'}\quad\textsf{(App1)}
\hspace{1cm}
\frac{e_0 \rightarrow e_0'}
     {e_0\ v \rightarrow e_0'\ v}\quad\textsf{(App2)}
\hspace{1cm}
%\frac{}
     {(\lambda x.e)\ v\rightarrow e\langle v/x\rangle}\quad\textsf{(Red)}
$$
$$
\frac{e_0 \rightarrow e_0'}
     {\mathtt{if}\ e_0\ \mathtt{then}\ e_1\ \mathtt{else}\ e_2
      \rightarrow \mathtt{if}\ e_0'\ \mathtt{then}\ e_1\ \mathtt{else}\ e_2
     }\quad\textsf{(Cond)}
$$

$$
\frac{v=\mathtt{true}}
     {\mathtt{if}\ v\ \mathtt{then}\ e_1\ \mathtt{else}\ e_2
      \rightarrow  e_1
     }\quad\textsf{(CondTrue)}
\hspace{1cm}
\frac{v=\mathtt{false}}
     {\mathtt{if}\ v\ \mathtt{then}\ e_1\ \mathtt{else}\ e_2
      \rightarrow  e_2
     }\quad\textsf{(CondFalse)}
$$

$$
     {\mathtt{rec}\ f\ x.e} \rightarrow \lambda x.e\langle \mathtt{rec}\ f\ x.e/f\rangle \quad\textsc{(Rec)}
$$

\vspace{0.1cm}
\hrule
\caption{\label{figsem}Operational semantics for our language.} 
\end{figure}


\section{Soundness of the Type System}
\label{correct}

In this section, we introduce a subject reduction theorem proving the consistency of our type system.
We use two operational semantics $\rightarrow_\bbf$ and $\rightarrow_\bbr$ for the finite precision
and exact arithmetics, respectively. 
The exact semantics is used for proofs. Obviously, in practice, only
the finite precision semantics is implemented.
We write $\rightarrow$ whenever a reduction rule holds for either
$\rightarrow_\bbf$ or $\rightarrow_\bbr$ (in this case, we assume that the same semantics $\rightarrow_\bbf$
or $\rightarrow_\bbr$ is used in the lower and upper parts of the same sequent).
The finite precision and exact semantics are  
displayed in Figure \ref{figsem}.
They concern the subset of the language of Equation (\ref{eqsyntax}) which do not deal with types
and defined by
 
\begin{equation}
\label{eqsyntax2}
\begin{array}{rcl}
\mathsf{EvalExpr}\ni e&::=&\ \mathtt{r}\{s,u,p\}\in \mathsf{Real}_{u,p}\ |\ \mathtt{i}\in \mathsf{Int}
\ |\ \mathtt{b}\in \mathsf{Bool}\ |\ \mathtt{id}\in\mathsf{Id} \\
&&|\  \mathtt{if}\ e_0\ \mathtt{then}\ e_1\ \mathtt{else}\ e_2
\ |\ \lambda x.e
\ |\ e_0\ e_1\ |\ \mathtt{rec}\ f\ x.e
|\ e_0\ \ast\ e_1
\end{array}\enspace .
\end{equation}

In Equation (\ref{eqsyntax2}), $\ast$ denotes an arithmetic operator
$\ast\in\{+,-,\times,\div,+\_,-\_,\times\_,\div\_\}$.
In Figure \ref{figsem}, Rule \textsc{(FVal)} of $\rightarrow_\bbf$ transforms a syntactic element 
describing a real number
$\mathtt{r\{s,u,p\} }$ in a certain format into a value $v_\bbf$.
The finite precision value $v_\bbf$ is an approximation of $\mathtt{r}$ with
an error less than the \textsf{ulp} of $\mathtt{r\{s,u,p\} }$.
In the semantics $\rightarrow_\bbr$, the real number $\mathtt{r\{s,u,p\} }$ simply
produces the value \texttt{r} without any approximation by Rule \textsf{(RVal)}.
Rules \textsf{(Op1)} and  \textsf{(Op2)} evaluate the expressions corresponding to the operands
of some binary operation and Rule  \textsf{(Op)} performs an operation $\ast\in\{+,-,\times,\div,+\_,-\_,\times\_,\div_\_\}$
between two values $v_0$ and $v_1$. 

Rules \textsf{(Cmp1)}, \textsf{(Cmp2)} and \textsf{(ACmp)} deal with comparisons.
They are similar to Rules \textsf{(Op1)}, \textsf{(Op2)} and \textsf{(Op)} described earlier.
Note that the operators $<,\ >,\ =,\ \not=$ concerned by Rule \textsf{(ACmp)} are polymorphic
excepted that they do not accept arguments of type \texttt{real}.
Rules \textsf{(FCmp)} and \textsf{(RCmp)} are for the comparison of \texttt{real} values. Rule 
\textsf{(FCmp)} is designed 
to avoid unstable tests by requiring that the distance between the 
two compared values is greater than the \textsf{ulp} of the format in which the comparison is
done. With this requirement, a condition cannot be invalidated by the roundoff errors.
Let us also note  that, with this definition, $x<_{u,p} y \not\Rightarrow y>_{u,p} x$
or $x>_{u,p} y \not\Rightarrow y<_{u,p} x$. For the semantics $\rightarrow_\bbr$, Rule \textsf{(RCmp)}
simply compares the exact values.

The other rules are standard and are identical in $\rightarrow_\bbf$
and $\rightarrow_\bbr$. Rules \textsf{(App1)}, \textsf{(App2) and \textsf{(Red)}}
are for applications and Rule \textsf{(Rec)} is for recursive functions. 
 We write $e\langle v/x\rangle$ the term $e$ in which $v$ has been substituted
to the free occurrences of $x$. Finally, Rules \textsf{(Cond)}, \textsf{(CondTrue)} and \textsf{(CondFalse)} 
are for conditionals.


\begin{figure}[tb]
\hrule
\vspace{0.1cm}
$$
\frac{}
     {\Gamma \models (\texttt{i},\texttt{i}) :\ \mathtt{int} }\quad\textsc{(Int)}
	 \hspace{1cm}
\frac{}
     {\Gamma \models (\texttt{b},\texttt{b}) :\ \mathtt{bool} }\quad\textsc{(Bool)}
\hspace{1cm}
\frac{\Gamma(\mathtt{id})=t}
     {\Gamma \models (\mathtt{id},\mathtt{id})\ :\ t }\quad\textsc{(Id)}
$$

$$
\frac{\mathsf{sign}(\texttt{r}) \prec s\quad \mathsf{ufp}(\texttt{r})\le u}
     {\Gamma \models (\texttt{r\{s,u,p\}},\texttt{r\{s,u,p\}})\ :\ \F{s}{u}{p} }\quad\textsc{(SReal)}
\hspace{0.5cm}
\frac{|v_\bbr-v_\bbf|<2^{u-p+1}}
     {\Gamma \models (v_\bbf,v_\bbr)\ :\ \F{s}{u}{p} }\quad\textsc{(VReal)}
$$

$$
\frac{\Gamma \models (e_{1\bbf},e_{1\bbr}) : \F{s_1}{u_1}{p_1}\hspace{0.5cm} 
      \Gamma \models (e_{2\bbf},e_{2\bbr}) : \F{s_1}{u_1}{p_1}
\hspace{0.5cm}\ast\in\{+,-,\times,\div\}
}
     {\Gamma \models (e_{1\bbf}\ast e_{2\bbf},e_{1\bbr}\ast e_{2\bbr}) :  
\F{\mathcal{S}_\ast(s_1,u_1,s_2,u_2)}{\mathcal{U}_\ast(s_1,u_1,s_2,u_2)}{\mathcal{P}_\ast(s_1,u_1,p_1,s_2,u_2,p_2)}
}\quad\textsc{(ROp)}
$$

$$
\frac{\Gamma \models (e_{1\bbf},e_{1\bbr}) : \F{s_1}{u}{p+1}\hspace{0.5cm} 
      \Gamma \models (e_{2\bbf},e_{2\bbr}) : \F{s_1}{u}{p+1}
\hspace{0.5cm}\ast\in\{<,>\}
}
     {\Gamma \models (e_{1\bbf}\Join_{u,p} e_{2\bbf},e_{1\bbr}\Join_{u,p} e_{2\bbr}) :  
\mathtt{bool}
}\quad\textsc{(RCmp)}
$$

$$
\frac{\Gamma \models (e_{1\bbf},e_{1\bbr}) : \mathtt{int}\hspace{0.5cm} 
      \Gamma \models (e_{2\bbf},e_{2\bbr}) : \mathtt{int}
\hspace{0.5cm}{\ast\_}\in\{+\_,-\_,\times\_,\div\_\}
}
     {\Gamma \models (e_{1\bbf}\ {\ast\_}\ e_{2\bbf},e_{1\bbr}\ {\ast\_}\ e_{2\bbr}) :  
\mathtt{int}
}\quad\textsc{(IntOp)}
$$

$$
\frac{\Gamma \models (e_{1\bbf},e_{1\bbr}) : t\hspace{0.5cm} 
      \Gamma \models (e_{2\bbf},e_{2\bbr}) : t\hspace{0.5cm}t\not=\F{s}{u}{p}
\hspace{1cm}\Join\in\{=,\not=,<,>,\le,\ge\}
}
     {\Gamma \models (e_{1\bbf}\Join e_{2\bbf},e_{1\bbr}\Join e_{2\bbr}) :  
\mathtt{bool}
}\quad\textsc{(ACmp)}
$$


$$
\frac{
\Gamma \models (e_{0\bbf},e_{0\bbr})\ :\ \mathtt{bool}\hspace{0.5cm}\Gamma \models (e_{1\bbf},e_{1\bbr})\ :\ t_1
\hspace{0.5cm} \Gamma \models (e_{2\bbf},e_{2\bbr})\ :\ t_2
\hspace{0.5cm} t=t_1\sqcup t_2}
{\Gamma\models (\mathtt{if}\ e_{0\bbf}\ \mathtt{then}\ e_{\bbf}1\ \mathtt{else}\ e_{2\bbf},
                \mathtt{if}\ e_{0\bbr}\ \mathtt{then}\ e_{1\bbr}\ \mathtt{else}\ e_{2\bbr})\ :\  t}\quad\textsc{(Cond)}
$$

$$
\frac{\Gamma,x :t_1 \models (e_\bbf,e_\bbr) :  t_2}
     {\Gamma \models (\lambda x.e_\bbf,\lambda x.e_\bbr) : \Pi x:t_1 .  t_2}\quad\textsc{(Abs)}
$$

$$
\frac{\Gamma \models (e_{1\bbf},e_{1\bbr}) : \Pi x: t_0. t_1\hspace{1cm} \Gamma \models (e_{2\bbf},e_{2\bbr}) :  t_2
\hspace{1cm} t_2\sqsubseteq t_0
}
     {\Gamma \models (e_{1\bbf}\ e_{2\bbf},e_{1\bbr}\ e_{2\bbr}) :  t_2[x\mapsto e_2]}\quad\textsc{(App)}
$$

$$
\frac{\Gamma,x :t_1,f:\Pi.y:t_1.t_2  \models (e_\bbf,e_\bbr) :  t_2}
     {\Gamma \models (\mathtt{rec}\ f\ x.e_\bbf,\mathtt{rec}\ f\ x.e_\bbr) : \Pi x:t_1 .  t_2}\quad\textsc{(Rec)}
$$
\vspace{0.1cm}
\hrule
\caption{\label{figmod}Inference rules for the simulation relation $\models$ used in our subject reduction theorem.} 
\end{figure}


The rest of this section is dedicated to our subject reduction theorem.
First of all, we need to relate the traces of $\rightarrow_\bbf$ and $\rightarrow_\bbr$.
We introduce new judgments 
\begin{equation}\label{eqmod}
\Gamma \models (e_\bbf,e_\bbr)\ :\ t\enspace . 
\end{equation}
Intuitively, Equation (\ref{eqmod}) means that  expression $e_\bbf$ simulates 
$e_\bbr$ up to accuracy $t$. In this case, $e_\bbf$
is syntactically equivalent to $e_\bbr$ up to the values which, in $e_\bbf$,
are approximations of the values of $e_\bbr$. The quantification of the approximation
is given by type $t$. 

Formally, $\models$ is defined in Figure \ref{figmod}. These rules are similar to the typing rules
rules of Figure \ref{figtyp} excepted that they operate on pairs $(e_\bbf,e_\bbr)$.
They are also designed for the language of Equation (\ref{eqsyntax2}) and, consequently,
deal with the elementary arithmetic operations $+,\ -,\ \times$ and $\div$ as well
as the comparison operators. 
The difference between the rules of Figure \ref{figtyp} and Figure \ref{figmod} is
in Rule \textsf{(VReal)} which states that a  \texttt{real} value $v_\bbr$
is correctly simulated by a value $v_\bbf$ up to accuracy $\F{s}{u}{p}$ if
$|v_\bbr-v_\bbf|<2^{u-p+1}$. 
It is easy to show, by examination of the rules of figures \ref{figtyp} and \ref{figmod}
that
\begin{equation}
\Gamma \models (e_\bbf,e_\bbr)\ :\ t\ \Longrightarrow\Gamma \vdash e_\bbf\ :\ t\enspace .
\end{equation}

We introduce now Lemma \ref{wsubject} which states the soundness of the type system for one
reduction step. 
Basically, this lemma states that types are preserved by reduction and that 
concerning the values of type \texttt{real}, the distance between the finite precision value
and the exact value is less than the \textsf{ulp} given by the format of the type.


\begin{lemma}[Weak subject reduction]
\label{wsubject}
If\ $\Gamma \models (e_\bbf,e_\bbr)\ : t$ and if  $e_\bbf\rightarrow_\bbf e_\bbf'$ and
$e_\bbr\rightarrow_\bbr e_\bbr'$ then $\Gamma \models (e_\bbf',e_\bbr')\ : t$. 
%In addition,
%if $e\equiv v$ and $t=\mathtt{real\{s,u,p\}}$ then $|\mathbb{R}(v)-\mathbb{F}(v)|< 2^{u-p+1}$.
\end{lemma} 

\begin{proof}
By induction on the structure of expressions
and case examination on the possible transition rules of Figure \ref{figsem}.
\begin{itemize}
%%%%%%%%%%%%
\item If $e_\bbf\equiv e_\bbr\equiv\mathtt{r\{s,u,p\}}$ then %, from  Rule \textsf{(SReal)} 
%of Figure \ref{figmod}, 
$\Gamma \models (\texttt{r\{s,u,p\}},\texttt{r\{s,u,p\}})\ :\ \F{s}{u}{p} $ and,
from the reduction rules \textsf{(FVal)} and \textsf{(RVal)} of Figure \ref{figsem},
$\mathtt{r\{s,u,p\}} \rightarrow_\bbf v_\bbf$ and
$\mathtt{r\{s,u,p\}} \rightarrow_\bbr v_\bbr$
with $|v_\bbf-v_\bbf|< 2^{u-p+1}$.
So $\Gamma\models (v_\bbf,v_\bbr)\ :\ \F{s}{u}{p} $.
%%%%%%%%%%%
\item If $e_\bbf\equiv e_{0\bbf}\ast e_{1\bbf}$ and $e_\bbr\equiv e_{0\bbr}\ast e_{1\bbr}$
 then several cases must be distinguished. 
\begin{itemize}
\item If $e_\bbf\equiv v_{0\bbf}\ \ast\ v_{1\bbf}$ and $e_\bbr\equiv v_{0\bbr}\ \ast\ v_{1\bbr}$ then, by induction hypothesis,
$\Gamma\models (v_{0\bbf},v_{0\bbr})\ :\mathtt{real\{s_0,u_0,p_0\}}$, 
$\Gamma\models (v_{1\bbf},v_{1\bbr})\ :\mathtt{real\{s_1,u_1,p_1\}}$ and, consequently, from
Rule \textsc{(VReal)},
\begin{equation}\label{eqproof2}
|v_{0\bbr}-v_{0\bbf}|<2^{u_0-p_0+1}\quad \text{and}\quad|v_{1\bbr}-v_{1\bbf}|<2^{u_1-p_1+1}\enspace. 
\end{equation}
Following Figure \ref{figtypprim}, the type $t$ of $e$ is
$$
\begin{array}{rcl}
t&=& \big(\Pi \mathtt{s_1}:\texttt{int},\mathtt{u_1}:\texttt{int}, \mathtt{p_1}:\texttt{int},
       \mathtt{s_2}:\texttt{int},\mathtt{u_2}:\texttt{int}, \mathtt{p_2}:\texttt{int}.\\
%\hspace{0.8cm}
&&\quad \F{s_1}{u_1}{p_1}\rightarrow\F{s_2}{u_2}{p_2}\rightarrow \\
&&\quad \rightarrow\F{\mathcal{S}_\ast(\mathtt{s_1},\mathtt{u_1},\mathtt{s_2},\mathtt{u_2})}
{\mathcal{U}_\ast(\mathtt{s_1},\mathtt{u_1},\mathtt{s_2},\mathtt{u_2})}
{\mathcal{P}_\ast(\mathtt{s_1},\mathtt{u_1},\mathtt{p_1},\mathtt{s_2},\mathtt{u_2},\mathtt{p_2})}\\
&&\big)\ \mathtt{s_1\ u_1\ p_1\ s_2\ u_2\ p_2}\enspace ,\\
&=&\F{\mathcal{S}_\ast(\mathtt{s_1},\mathtt{u_1},\mathtt{s_2},\mathtt{u_2})}
{\mathcal{U}_\ast(\mathtt{s_1},\mathtt{u_1},\mathtt{s_2},\mathtt{u_2})}
{\mathcal{P}_\ast(\mathtt{s_1},\mathtt{u_1},\mathtt{p_1},\mathtt{s_2},\mathtt{u_2},\mathtt{p_2})}\\
&=&\mathtt{real\{s,u,p\}}
\end{array}
$$
By Rule \textsc{(Op)}, $e\rightarrow_\bbf v_\bbf$ and $e\rightarrow_\bbr v_\bbr$ and, 
by Theorem \ref{thop}, with  the assumptions of Equation (\ref{eqproof2}), we know that $|v_\bbr-v_\bbf|<2^{u-p+1}$.
Consequently, $\Gamma\models (v_\bbf,v_\bbr)\ :\ \F{s}{u}{p}$.
\item If $e_\bbf\equiv v_{0\bbf}\ \ast\ v_{1\bbf}$ and $e_\bbr\equiv v_{0\bbr}\ \ast\ v_{1\bbr}$
 with $\Gamma\models( v_0, v_1)\ :\ \mathtt{int}$ then,
by Rule \textsf{(Op)}, $e\rightarrow (v,v)$ and,
by Equation (\ref{eqtypint}), $\Gamma\vdash v\ :\ \mathtt{int}$.
If $e\equiv e_0\ast e_1$ then, by Rule $\textsf{(Op1)}, e\rightarrow e_0\ast e_1'$ and we conclude by
induction hypothesis. The case $e\equiv e_0\ast\ v_1$ is similar to the former one.
\end{itemize}
%%%%%%%%%%%%%%%%%%%%%%%%%%%%%%%%%%%%%%%%%
\item If $e_\bbf\equiv e_{0\bbf}\Join_{u,p} e_{1\bbf}$ and $e_\bbr\equiv e_{0\bbr}\Join_{u,p} e_{1\bbr}$
then several cases have to be examined.
\begin{itemize}
\item If $e_\bbf\equiv v_{0\bbf}\Join_{u,p} v_{1\bbf}$ and
$e_\bbr\equiv v_{0\bbr}\Join_{u,p} v_{1\bbr}$
 then by rules \textsf{(FCmp)} and \textsf{(RCmp)}
$e_\bbf\rightarrow_\bbf b_\bbf$, $e_\bbr\rightarrow_\bbr b_\bbr$ with
\begin{equation}\label{eqproofjoin}
b_\bbf=     v_{0\bbf}- v_{1\bbf} \Join_{\{u,p\}} 2^{u-p+1}
\quad\text{and}\quad
b_\bbf=     v_{0\bbr} -v_{1\bbr} \Join_{\{u,p\}} 0 \enspace .
\end{equation}
By rule \textsf{(RCmp)} of Figure \ref{figmod}, $\Gamma\models (v_{0\bbf},v_{1\bbf})\ :\ \F{s}{u}{p}$
and $\Gamma\models (v_{0\bbr},v_{1\bbr})\ :\ \F{s}{u}{p}$.
Consequently,
\begin{equation}
|v_{0\bbr}-v_{0\bbf}| < 2^{u-p+1}\quad\text{and}\quad |v_{1\bbr}-v_{1\bbf}| < 2^{u-p+1}
\end{equation}
By combining equations (\ref{eqproofjoin}) and (\ref{eqproofjoin}), we obtain that
\begin{equation}
|(v_{0\bbr}-v_{1\bbr})-(v_{0\bbf}-v_{1\bbf})| < 2^{u-p}\enspace .
\end{equation}
Consequently, $b_\bbf=b_\bbr$ and we conclude that $\Gamma\models (b_\bbf,b_\bbr)\ : \mathtt{bool}$.
\item The other cases for $e_\bbf\equiv e_{0\bbf}\Join_{u,p} e_{1\bbf}$
are similar to the cases $e_\bbf\equiv v_{0\bbf}\ast v_{1\bbf}$ examined previously.
\end{itemize}
\item The other cases simply follow the structure of the terms, by application of the induction hypothesis.
\end{itemize}
\end{proof}



Let $\rightarrow^*$ denote the reflexive transitive closure of $\rightarrow$.
Theorem \ref{subject} expresses the soundness of our type system for sequences of reduction
of arbitrary length.

\begin{theorem}[Subject reduction]
\label{subject}
If\ $\Gamma \vdash e\ : t$ and $e\rightarrow^* e'$ then $\Gamma \vdash e'\ : t$. In addition,
if $e\equiv v$ and $t=\mathtt{real\{s,u,p\}}$ then $|\mathbb{R}(v)-\mathbb{F}(v)|< 2^{u-p+1}$.
\end{theorem} 

\begin{proof}
By induction on the length of the reduction sequence, using Lemma \ref{wsubject}.
\end{proof}

Theorem \ref{subject} assert the soundness of our type system. It states that the evaluation of an expression of type $\F{s}{u}{p}$ yields
a result of accuracy $2^{u-p+1}$.


