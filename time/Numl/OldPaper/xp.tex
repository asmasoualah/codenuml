
\subsection{Experiments}
\label{morex}

In this section, we report some experiments showing how our type system behaves in practice.

\subsubsection{Usual Mathematic Formulas}
\label{sssusmath}

Our first examples concern usual mathematic formulas, to compute the volume of 
geometrical objects or formulas related to polynomials. 
These examples aim at showing that usual mathematical formulas are typable in our system.
We start with the volume of the sphere and of the cone.

\begin{verbatim}
> let sphere r = (4.0 / 3.0) * 3.1415926{+,1,20} * r * r * r ;;
val sphere : real{'a,'b,'c} -> real{<expr>,<expr>,<expr>} = <fun>

> sphere 1.0 ;;
- : real{+,7,20} = 4.188

> let cone r h = (3.1415926{+,1,20} * r * r * h) / 3.0 ;;
val cone : real{'a,'b,'c} -> real{'a,'b,'c} 
           -> real{<expr>,<expr>,<expr>} = <fun>

> cone 1.0 1.0 ;;
- : real{+,4,20} = 1.0472
\end{verbatim}

We repeatedly define the function \texttt{sphere}  with more precision in order
to show the impact on the accuracy of the results. Note that the results now have $15$ digits instead
of the former $5$ digits.

\begin{verbatim}
> let sphere r = (4.0 / 3.0) * 3.1415926535897932{+,1,53} * r * r * r ;;
val sphere : real{'a,'b,'c} -> real{<expr>,<expr>,<expr>} = <fun>

> sphere 1.0 ;; 
- : real{+,7,52} = 4.1887902047863
\end{verbatim}


The next examples concern polynomials. We start with 
 the computation of the discriminant of a second degree polynomial.

\begin{verbatim}
> let discriminant a b c = b * b - 4.0 * a * c ;;
val discriminant : real{'a,'b,'c} -> real{'d,'e,'f} -> real{'g,'h,'i} 
                   -> real{<expr>,<expr>,<expr>} = <fun>

> discriminant 2.0 -11.0 15.0 ;;
- : real{+,8,52} = 1.000000000000
\end{verbatim}

Our last example concerning usual formulas is the Taylor series development of the sine function. In the
code below, observe
that the accuracy of the result is correlated to the accuracy of the argument.
As mentioned in Section \ref{over}, error methods are neglected, only the errors due
to the finite precision are calculated (indeed, $\sin \frac{\pi}{8}=0.382683432\ldots$).

\begin{verbatim}
let sin x = x - ((x * x * x) / 3.0) + ((x * x * x * x * x) / 120.0) ;;
val sin : real{'a,'b,'c} -> real{<expr>,<expr>,<expr>} = <fun>

> sin (3.14{1,6} / 8.0) ;;
- : real{*,0,6} = 0.3

> sin (3.14159{1,18} / 8.0) ;;
- : real{*,0,18} = 0.37259
\end{verbatim}


\subsubsection{Elementary Functions}

We introduce hereafter some implementations of elementary mathematical functions.
Contrarily to the examples of Section \ref{sssusmath},
the computation of these new functions gives rise to simple recursions.
Our first example implements the Taylor series
\begin{equation}
\frac{1}{1-x}=\sum_{n\ge 0} x^n
\end{equation}
We have

\begin{verbatim}
> let rec taylor x{-1,25} xn i n = if (i > n) then 0.0{*,10,20}
                                   else xn + (taylor x (x * xn) (i +_ 1) n) ;;

val taylor : real{*,-1,25} -> real{*,10,20} -> int -> int -> real{*,10,20} = <fun>

> taylor 0.2 1.0 0 5;;
- : real{*,10,20} = 1.2499 +/- 0.0009765625
\end{verbatim}

Our second example is an implementation of the square root function.

\begin{verbatim}

\end{verbatim}



